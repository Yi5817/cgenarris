%% LyX 2.3.2 created this file.  For more info, see http://www.lyx.org/.
%% Do not edit unless you really know what you are doing.
\documentclass[english]{extarticle}
\usepackage[latin9]{inputenc}
\usepackage{geometry}
\geometry{verbose}
\usepackage{graphicx}

\makeatletter

%%%%%%%%%%%%%%%%%%%%%%%%%%%%%% LyX specific LaTeX commands.
%% A simple dot to overcome graphicx limitations
\newcommand{\lyxdot}{.}


\makeatother

\usepackage{babel}
\begin{document}
\title{Cgenarris Documentation}
\author{Marom group}

\maketitle
\newpage

\tableofcontents{}

\newpage

\section{Features}
\begin{itemize}
\item Support for special positions.
\item Parallelized using OpenMP.
\item Fast and efficient structure checking.
\item Can generate in all possible Z (or NMPC) with Z' \textless = 1.
\end{itemize}
\newpage

\section{Installation}

\subsection{Cgenarris}

\subsubsection*{Requirements}

Any C compiler which supports ANSI C99 / GNU99 standard.

\subsubsection*{Using Makefile}
\begin{enumerate}
\item Uncompress the tar file.
\item Execute 'make cgenarris'.
\item This will create cgenarris.x which is the desired executable.
\end{enumerate}
NOTE:
\begin{enumerate}
\item You may change the C compiler using the environment variable CC. You
may also uncomment the first line of the makefile and set the compiler.
\item Remove object files using 'make clean'
\end{enumerate}

\subsection{Pygenarris}

\subsubsection*{Requirements}
\begin{enumerate}
\item Any C compiler which supports ANSI C99 / GNU99 standard.
\item SWIG (Simplified Wrapper code and Interface Generator).
\item Numpy
\item Distutils for installation through setup.py
\end{enumerate}

\subsubsection*{Method 1: Using Makefile}
\begin{enumerate}
\item Uncompress the tar file.
\item Paste the location of Python.h headerfile in the Makefile. (for Anaconda
v3.7 it should be ' anaconda/include/python3.7m/ ')
\item Execute 'make pygenarris'.
\item This will create pygenarris.so library from which you can import pygenarris.
\end{enumerate}

\subsubsection*{Method 2: Using Distutils}
\begin{enumerate}
\item Uncompress the tar file.
\item Execute ' python setup.py build\_ext -{}-inplace'.
\item This will create pygenarris.so library from which you can import pygenarris.
\end{enumerate}
NOTE
\begin{enumerate}
\item You may change the C compiler using the environment variable CC.
\item pygenarris can be configured with both python2 and python3.
\end{enumerate}
\newpage

\section{Pygenarris}

Pygenarris is a python API for C structure generator and associated
functions.

\subsection{Generate a pool of random molecular crystals}

\emph{generate\_molecular\_crystals(filename, num\_structures, Z,
volume\_mean, volume\_std, sr, tol, max\_attempts)}

\subsubsection*{Description}

Generate random molecular crystals by space groups. First, the generator
first identifies space groups that are compatible with molecular symmetry
and given number of molecules in the unit cell. Structures are generated
sequentially from lowest space group to the highest. Cell volumes
are sampled from a normal distribution. The attempted structures are
checked for closeness of molecules. If an atom of a molecule is too
close to its own periodic image or another atom of a different molecule
in a cell, the structure is discarded. The closeness checks are controlled
by the specific radius proportion (sr). If the generation of a space
group fails after max\_attempts times, the generator moves to the
next higher space group. The generated structures are printed to the
file in FHI-aims geometry.in format.

\subsubsection*{Input}
\begin{enumerate}
\item Geometry of the molecule is read from the file \emph{geometry.in }from
the working directory.
\item \emph{filename }is the name of the file to which generated structures
are printed. Type: string
\item \emph{num\_structures }is the number of structures from each space
group. Type: integer
\item \emph{Z }number of molecules in the conventional cell. Type: integer
\item \emph{volume\_mean }is the mean of the normal distribution from which
volume is sampled. Type: float
\item \emph{volume\_std }is the standard deviation of the volume distribution.
Type: float
\item \emph{sr }is specific radius proportion. See Genarris paper for definition.
Type: float
\item \emph{tol} is the tolerance for special position generation and space
group detection. Type: float
\item \emph{max\_attempts }is the maximum number of attempts before moving
to the next space group. Type: integer
\item The number of threads can be set by using the environment variable
\emph{OMP\_NUM\_THREADS.}
\end{enumerate}

\subsubsection*{Output}
\begin{enumerate}
\item A file with all the generated structures in FHI-aims geometry format.
\end{enumerate}

\subsection{Identification of compatible space groups given molecular symmetry}

\emph{find\_allowed\_positions\_using\_molecular\_symmetry(point\_group,
Z, Z'')}

\subsubsection*{Description}

This function finds the compatible space group positions using molecule's
point group.

\subsubsection*{Input}
\begin{enumerate}
\item \emph{point\_group }is the point\_group of the molecule. Eg: ``mmm''
for tetracene. Type: String.
\item \emph{Z }is the number of molecules in the conventional cell. Type:
integer
\item \emph{Z'' }is the number of inequivalent molecules in the cell. \emph{Not
implemented! Set any integer.}
\end{enumerate}

\subsubsection*{Output}
\begin{enumerate}
\item Compatible Wyckoff positions and space groups are printed.
\end{enumerate}

\subsubsection*{Example}

\includegraphics[scale=0.4]{find_allowed_positions\lyxdot png}


\end{document}
